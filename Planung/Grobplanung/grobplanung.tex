% !TEX TS-program = pdflatex
% !TEX encoding = UTF-8 Unicode

% This is a simple template for a LaTeX document using the "article" class.
% See "book", "report", "letter" for other types of document.

\documentclass[11pt]{article} % use larger type; default would be 10pt

\usepackage[utf8]{inputenc} % set input encoding (not needed with XeLaTeX)

%%% Examples of Article customizations
% These packages are optional, depending whether you want the features they provide.
% See the LaTeX Companion or other references for full information.

%%% PAGE DIMENSIONS
\usepackage{geometry} % to change the page dimensions
\geometry{a4paper} % or letterpaper (US) or a5paper or....
% \geometry{margin=2in} % for example, change the margins to 2 inches all round
% \geometry{landscape} % set up the page for landscape
%   read geometry.pdf for detailed page layout information

\usepackage{graphicx} % support the \includegraphics command and options

% \usepackage[parfill]{parskip} % Activate to begin paragraphs with an empty line rather than an indent

%%% PACKAGES
\usepackage{booktabs} % for much better looking tables
\usepackage{array} % for better arrays (eg matrices) in maths
\usepackage{paralist} % very flexible & customisable lists (eg. enumerate/itemize, etc.)
\usepackage{verbatim} % adds environment for commenting out blocks of text & for better verbatim
\usepackage{subfig} % make it possible to include more than one captioned figure/table in a single float
% These packages are all incorporated in the memoir class to one degree or another...

%%% OWN
\usepackage{tabto}

%%% HEADERS & FOOTERS
\usepackage{fancyhdr} % This should be set AFTER setting up the page geometry
\pagestyle{fancy} % options: empty , plain , fancy
\renewcommand{\headrulewidth}{0pt} % customise the layout...
\lhead{}\chead{}\rhead{}
\lfoot{}\cfoot{\thepage}\rfoot{}

%%% SECTION TITLE APPEARANCE
\usepackage{sectsty}
\allsectionsfont{\sffamily\mdseries\upshape} % (See the fntguide.pdf for font help)
% (This matches ConTeXt defaults)

%%% ToC (table of contents) APPEARANCE
\usepackage[nottoc,notlof,notlot]{tocbibind} % Put the bibliography in the ToC
\usepackage[titles,subfigure]{tocloft} % Alter the style of the Table of Contents
\renewcommand{\cftsecfont}{\rmfamily\mdseries\upshape}
\renewcommand{\cftsecpagefont}{\rmfamily\mdseries\upshape} % No bold!

%%% END Article customizations

%%% The "real" document content comes below...

\title{Grobplanung}
\author{Renato Bosshart, Josua Schmid}
%\date{} % Activate to display a given date or no date (if empty),
         % otherwise the current date is printed 

\lhead{SA Kinect}
\chead{Renato Bosshart, Josua Schmid}
\rhead{\today}

\begin{document}
\maketitle

\section{Wochenplanung}

\begin{itemize}
\setlength{\itemsep}{-10pt}
	\item W1: \tabto{15mm} Einarbeitung und Aufsetzen der Umgebung \\
	\item W2: \tabto{15mm} Evaluation bestehender Projekte \\
	\item W3: \tabto{15mm} Evaluation möglicher Bedienkonzepte \\
	\item W4-6: \tabto{15mm} Umsetzungsversuche und Tests der Bedienkonzepte \\
	\item W7: \tabto{15mm} Bedienkonzept wird auf Industrieanwendung fixiert.
			\\\tabto{15mm} Evaluation Konflikte \& Probleme \\
	\item W8-10: \tabto{15mm} Verbesserung der Inputs bezüglich Stabilität \\
	\item W11: \tabto{15mm} Physikalische Grenzwertanalyse (Technische Evaluation) \\
	\item W12: \tabto{15mm} Dokumente finalisieren \\
	\item W15: \tabto{15mm} Präsentation vorbereiten \\
	\item W14: \tabto{15mm} Präsentation
\end{itemize}

\section{Detaillierte Aufgabenbeschreibung}
\subsection{Projektevaluation}
Was gibt es für Kinect-Projekte? Was haben sie für Möglichkeiten? Wie funktionieren sie? OpenSource? Beispiele: FAAST, Mausemulation

\subsection{Bedienkonzepte}
Was für Konzepte bestehender Projekte? Wii, Kinect-Games, Lightgun, Xbox, Mausemulation, PS3, The Leap, Konventionelle Touchkonzepte.\\
Gibt es für unser Projekt spezielle Einschränkungen in der Bedienung? z.B.: Mitarbeiter sollen nicht vor dem zu bedienenden Gerät herumtanzen müssen.

\subsection{Technische Evaluation}
Welche physikalischen Limitierungen hat die Kinect? Wie kann man die Erkennung verbessern? z.B: Idealabstand der zu scannenden Objekte, Fremdlicht, Spiegel

\subsection{Probleme}
Wie sieht das Umfeld aus? Andere Personen dürfen nicht ablenken. Welche Person ist am Bedienen? Ist sie definitiv weg? Oder nur kurz?

\end{document}
